\documentclass[uplatex]{jsarticle}
\usepackage[utf8]{inputenc}

\usepackage{amssymb}
\usepackage{amsmath}
\usepackage{amsthm}
\usepackage{framed}
\usepackage{braket}
\usepackage{bm}
\usepackage{mathrsfs}
\usepackage{accents}
\usepackage{tocloft}
\usepackage[dvipdfmx]{graphicx}
\usepackage{tikz}
\usepackage{url}
\usepackage{color}
\usepackage{xifthen}
\usepackage{xcolor}
\usepackage{framed}
\usepackage{mathtools}
\usepackage[explicit]{titlesec}
\usepackage{mdframed}
\usepackage{geometry}
\geometry{left=30mm,right=30mm,top=20mm,bottom=20mm}
\usepackage{enumerate}
\usepackage[dvipdfmx]{hyperref}
\usepackage{pxjahyper}
\renewcommand{\baselinestretch}{1.2}

\usetikzlibrary{positioning}
\usetikzlibrary{calc}
\usetikzlibrary{decorations.pathreplacing}
\usetikzlibrary{cd}


\renewcommand{\labelenumi}{(\arabic{enumi})}

\newcommand{\scrN}{\mathcal{N}}
\newcommand{\scrI}{\mathcal{I}}
\newcommand{\scrC}{\mathcal{C}}
\newcommand{\scrJ}{\mathcal{J}}
\newcommand{\N}{\mathbb{N}}
\newcommand{\Z}{\mathbb{Z}}
\renewcommand{\P}{\mathbb{P}}
\newcommand{\B}{\mathbb{B}}
\newcommand{\Q}{\mathbb{Q}}
\newcommand{\R}{\mathbb{R}}
\newcommand{\C}{\mathbb{C}}
\newcommand{\range}{\operatorname{ran}}
\newcommand{\dom}{\operatorname{dom}}
\newcommand{\append}{{}^\frown}
\newcommand{\boldsig}{\boldsymbol{\Sigma}}
\newcommand{\boldpi}{\boldsymbol{\Pi}}
\newcommand{\bolddelta}{\boldsymbol{\Delta}}
\newcommand{\Ordinals}{\mathrm{On}}
\newcommand\forces{\Vdash}
\newcommand\notforces{\nVdash}
\newcommand{\cl}{\operatorname{cl}}
\newcommand{\intr}{\operatorname{int}}
\newcommand{\ro}{\operatorname{ro}}
\newcommand{\rank}{\operatorname{rank}}
\newcommand{\frakt}{\mathfrak{t}}
\newcommand{\s}{\mathfrak{s}}
\newcommand{\frakb}{\mathfrak{b}}
\newcommand{\frakd}{\mathfrak{d}}
\newcommand{\frakc}{\mathfrak{c}}
\newcommand{\Pow}{\mathcal{P}}
\newcommand{\non}{\operatorname{non}}
\newcommand{\cov}{\operatorname{cov}}
\newcommand{\add}{\operatorname{add}}
\newcommand{\cof}{\operatorname{cof}}
\newcommand{\Cof}{\mathbf{Cof}}
\newcommand{\Cov}{\mathbf{Cov}}
\newcommand{\D}{\mathbf{D}}
\newcommand{\Lc}{\mathbf{Lc}}
\newcommand{\nul}{\mathsf{null}}
\newcommand{\meager}{\mathsf{meager}}
\newcommand{\id}{\mathrm{id}}
\newcommand{\diam}{\mathrm{diam}}
\newcommand{\height}{\mathrm{ht}}
\newcommand{\pow}{\mathrm{pow}}
\newcommand{\GTle}{\preceq_\mathrm{GT}}
\newcommand{\Map}[2]{\operatorname{Map}(#1, #2)}
\newcommand{\omegaupomega}{\omega^{\uparrow \omega}}
\newcommand{\twototheltomega}{2^{<\omega}}
\newcommand{\cf}{\operatorname{cf}}
\newcommand{\LangL}{\mathcal{L}}
\newcommand{\Add}{\operatorname{Add}}
\newcommand{\Seq}{\operatorname{Seq}}
\newcommand{\stem}{\operatorname{stem}}
\newcommand{\suc}{\operatorname{succ}}
\newcommand{\Lev}{\operatorname{Lev}}
\newcommand{\AND}{\mathbin{\&}}
\newcommand{\OR}{\text{ or }}
\newcommand{\restrict}{\upharpoonright}
\newcommand{\Lim}{\mathrm{Lim}}
\newcommand{\Limone}{\mathrm{Lim}_{\omega_1}}
\newcommand{\ZFC}{\mathsf{ZFC}}
\newcommand{\CH}{\mathsf{CH}}
\newcommand{\subsetic}{\subseteq_{\mathrm{ic}}}
\DeclareMathOperator*{\diagintr}{\triangle}

\newcommand{\seq}[1]{{\langle#1\rangle}}
\DeclarePairedDelimiter\abs{\lvert}{\rvert}
\DeclarePairedDelimiter\floor{\lfloor}{\rfloor}
\DeclarePairedDelimiter\ceil{\lceil}{\rceil}

\renewcommand\emptyset{\varnothing}
\renewcommand\subset{\subseteq}
\renewcommand{\setminus}{\smallsetminus}

\newcommand{\needtocheck}[1][]{%
	\ifthenelse{\equal{#1}{}}{%
		\textcolor{blue}{[NeedToCheck]}%
	}{%
		\textcolor{blue}{[NeedToCheck: #1]}%
	}%
}

\newcommand{\todo}[1][]{%
	\ifthenelse{\equal{#1}{}}{%
		\textcolor{red}{[TODO]}%
	}{%
		\textcolor{red}{[TODO: #1]}%
	}%
}


\theoremstyle{definition}
\newtheorem{thm}{定理}[section]
\newtheorem*{thm*}{定理}
\newtheorem{defi}[thm]{定義}
\newtheorem*{defi*}{定義}
\newtheorem{lem}[thm]{補題}
\newtheorem*{lem*}{補題}
\newtheorem{fact}[thm]{事実}
\newtheorem*{fact*}{事実}
\newtheorem{prop}[thm]{命題}
\newtheorem*{prop*}{命題}
\newtheorem{exm}[thm]{例}
\newtheorem*{exm*}{例}
\newtheorem{rmk}[thm]{注意}
\newtheorem*{rmk*}{注意}
\newtheorem{cor}[thm]{系}
\newtheorem*{cor*}{系}
\newtheorem*{notation*}{記法}
\newtheorem{asm}[thm]{仮定}
\newtheorem{prob}[thm]{問題}
\newtheorem{conj}[thm]{予想}
\renewcommand{\proofname}{証明}

\newenvironment{claim}[1]{\par\noindent\underline{主張 #1:}\space}{}
\newenvironment{claimproof}[1]{\par\noindent$\because$) \space#1}{\hfill //}



\usepackage[backend=biber,style=alphabetic,sorting=nty,doi=false,isbn=false,url=false,eprint=true]{biblatex}
\addbibresource{oracle-cc.bib}
\renewbibmacro{in:}{}


\title{\vspace{-2cm} \HUGE 神託連鎖条件強制法}
\author{でぃぐ}

\begin{document}
	\maketitle
	
	%\begin{abstract}
	%	hogehoge
	%end{abstract}
	
	\tableofcontents
	
	\section{神託連鎖条件強制法}
	
	\begin{defi}
		$\bar{M}$が\textbf{$\aleph_1$-神託}であるとは次を満たすことである:
		\begin{enumerate}
			\item $\bar{M}$は列$\bar{M} = \seq{M_\delta : \delta \in \Limone }$である.
			\item 各$M_\delta$は$\ZFC^-$の可算推移的モデルである.
			\item 各$\delta \in \Limone$について$\delta + 1 \subset M_\delta$かつ$M_\delta \models ``\delta\text{は可算}"$.
			\item 任意の$A \subset \omega_1$について$\{\delta \in \Limone : A \cap \delta \in M_\delta \}$は$\omega_1$の定常集合.
		\end{enumerate}
	\end{defi}
	
	\begin{lem}\label{lem:diamondimpliesoracles}
		ダイヤモンド原理$\diamondsuit$から$\aleph_1$-神託の存在が導かれる.
	\end{lem}
	\begin{proof}
		ダイヤモンド列$\seq{A_\alpha : \alpha < \omega_1}$をとる.
		%各$\delta \in \Limone$について全単射$f_\delta \colon \omega \to \delta$をとる.
		%$H_{\omega_1}$の,集合$(\delta + 1) \cup \{f_\delta\} \cup \{A_\delta\}$を含む可算初等部分モデル$N_\delta$を取り$N_\delta$の推移崩壊$M_\delta$とする.
		$H_{\omega_1}$の集合$(\delta + 1) \cup \{A_\delta\}$を含む可算初等部分モデル$N_\delta$を取り$N_\delta$の推移崩壊$M_\delta$とする.
		このとき$\seq{M_\delta : \delta \in \Limone }$が$\aleph_1$-神託である.
	\end{proof}

	補題\ref{lem:diamondimpliesoracles}の逆も正しいが,本稿では使わない.\cite{kunen1983set}のTheorem 7.14を参照せよ.

	\begin{defi}
		各$\aleph_1$-神託 $\bar{M}$に対して,$\omega_1$上のフィルター$D_{\bar{M}}$を集合たち
		\[
		I_{\bar{M}}(A) = \{ \delta \in \Limone : A \cap \delta \in M_\delta \} \text{ (for $A \subset \omega_1)$}
		\]
		で生成されるものとする.
	\end{defi}

	\begin{lem}
		\begin{enumerate}
			\item 任意の$A, B \subset \omega_1$について,$C \subset \omega_1$が存在して
			\[
			I_{\bar{M}}(C) = I_{\bar{M}}(A) \cap I_{\bar{M}}(B).
			\]
			\item $D_{\bar{M}}$は任意のclub集合であって$\Limone$に含まれるものを持つ.
			\item $D_{\bar{M}}$は真の正規フィルター.
		\end{enumerate}
	\end{lem}
	\begin{proof}
		(1) $A, B \subset \omega_1$を取る.$g, f \colon \omega_1 \to \omega_1$を
		\begin{align*}
		g(\alpha) &= 2 \alpha \\
		f(\alpha) &= 2 \alpha + 1
		\end{align*}
		で定める.
		$\delta < \omega_1$が極限順序数ならば$\delta$は$g$と$f$で閉じている.そこで絶対性により$g \restrict \delta, f \restrict \delta \in M_\delta$である.
		
		$C = g(A) \cup f(B)$とおく.
		すると
		\[
		\delta \in I_{\bar{M}}(C) \iff \delta \in I_{\bar{M}}(A) \AND \delta \in I_{\bar{M}}(B)
		\]
		を得る.
		
		(2) $C \subset \Lim_{\omega_1}$をclub集合とする.
		$\seq{\delta_i : i < \omega_1}$を単調増加で連続な$C$の枚挙とする.
		$A \subset \omega_1$であって次を満たすものを構成する:$\delta \in \Limone$かつすべての$i < \omega_1$に対して$\delta \ne \delta_i$ならば,$A \cap \delta \not \in M_\delta$.
		この$A$を構成し終えると
		\begin{align*}
			I_{\bar{M}}(A) &= \{ \delta \in \Limone : A \cap \delta \in M_\delta \} \\
			& \subset \{ \delta \in \Limone : \delta = \delta_i \text{ for some $i<\omega_1$}  \} \\
			& = C
		\end{align*}
		となるので$C \in D_{\bar{M}}$を得ることになる.
		
		$A$を区間ごとに帰納的に構成する.
		つまり$A \cap [\delta_i, \delta_i + \omega)$を$i < \omega_1$に関する帰納法で定めていく.
		これらの区間の外の順序数については必ず$A$に入れることにする.
		
		$A \cap \delta_i$が定まったとき,$2^{\aleph_0}$個の$A \cap (\delta_i + \omega)$の可能性がある.
		その中から一つ選び,可算集合
		\[
		\{ B \cap (\delta_i + \omega) : B \in M_\delta, \delta_i < \delta < \delta_{i+1} \}
		\]
		に属さないものとする.
		この構成で欲しい$A$が得られる.構成より$\delta \ne \delta_i$ for all $i < \omega_1$なる$\delta \in \Limone$に対して$A \cap \delta \not \in M_\delta$であるからだ.
		
		(3) 真のフィルターであることは(1)と各$A \subset \omega_1$について$I_{\bar{M}}(A)$が定常集合である,特に非空であることという事実から従う.
		
		正規性を示そう.
		対角共通部分で閉じていることを示す.
		それを示す際,とってくる元たちはフィルターの生成元であるとしてよいので,$I_{\bar{M}}(A_i)$ (各$i < \omega_1$について$A_i \subset \omega_1$)が与えられることとなる.$A \subset \omega_1$であって
		\[
		I_{\bar{M}}(A) \subset \diagintr_{i < \omega_1} I_{\bar{M}}(A_i)
		\]
		となるものを構成すればよい.つまり
		\[
		(\forall \delta \in \Limone)[A \cap \delta \in M_\delta \rightarrow (\forall i < \delta)(A_i \cap \delta \in M_\delta)]
		\]
		を言う.(2)と(3)よりclub manyな$\delta$についてこの式が言えれば良い.
		
		$\langle -, -\rangle : \omega_1 \times \omega_1 \to \omega_1$を十分良く定義されたペア関数とする.
		\[
		C = \{ \delta < \omega_1 : \text{$\delta$は$\langle -, -\rangle$で閉じている}\}
		\]
		とおけば$C$はclubである.
		$\delta \in C$について$\langle -, -\rangle$の$\delta \times \delta$への制限は絶対性より$M_\delta$に属する.
		\[
		A = \{ \langle i, \alpha\rangle : \alpha \in A_i \AND i < \omega_1 \}
		\]
		とおく.
		$\delta \in C$かつ$A \cap \delta \in M_\delta$を仮定し,$i < \delta$とする.
		このとき,$\langle -, -\rangle$で第一座標が$i$なものを取り出す関数は絶対的なことから$A_i \cap \delta \in M_\delta$を得る.これで示せた.
	\end{proof}

	\begin{lem}\label{lem:closure}
		$h \colon \Pow(\omega_1) \to \Pow(\omega_1)$を関数とする.このとき$\{ \delta \in \Limone : (\forall A \in M_\delta \cap \Pow(\delta))(h(A) \cap \delta \in M_\delta) \}$は$D_{\bar{M}}$の元である.
	\end{lem}
	\begin{proof}
			集合$\bigcup_{\delta \in \Limone} M_\delta \cap \Pow(\delta)$を$\bigcup_{\delta \in \Limone} M_\delta \cap \Pow(\delta) = \{ X_\alpha : \alpha \in \omega_1 \}$と枚挙する.
			ただしあるclub $C$について,どの$\delta \in C$に対しても$M_\delta \cap \Pow(\delta)$の元はすべて$\delta$未満の番号$\alpha$に対する$X_\alpha$として出現するようにする.このような枚挙は適当なペア関数を使ってbookkeepingをすれば可能である.
			$\alpha < \omega_1$に対して$Y_\alpha = I_{\bar{M}}(h(X_\alpha))$とおく.
			このとき
			\begin{align*}
				D_{\bar{M}} \ni C \cap \diagintr_{\alpha < \omega_1} Y_\alpha &= \{ \delta \in C : (\forall \alpha < \delta)(\delta \in Y_\alpha) \} \\
			 	&= \{ \delta \in C : (\forall \alpha < \delta)(\delta \in I_{\bar{M}}(h(X_\alpha))) \} \\
			 	&\supseteq \{ \delta \in C : (\forall A \in M_\delta \cap \Pow(\delta))(\delta \in I_{\bar{M}}(h(A)) \} \\
			 	&= \{ \delta \in C : (\forall A \in M_\delta \cap \Pow(\delta))(h(A) \cap \delta \in M_\delta) \}
			\end{align*}
			となる.これでよい.
		\end{proof}

	\begin{defi}\label{def:oraclecc}
		$\bar{M}$を$\aleph_1$-神託とする.強制概念$P$が\textbf{$\bar{M}$連鎖条件}を満たすとは,次の\.い\.ず\.れ\.かを満たすときである.
		\begin{enumerate}
			\item $\abs{P} \le \aleph_0$である.
			\item $\abs{P} = \aleph_1$かつある単射$f \colon P \to \omega_1$について
			\begin{align*}
			\{ \delta \in \Limone : &\text{集合}A\text{が}A \in M_\delta, A \subset \delta, \text{$f^{-1}(A)$が前稠密 in $f^{-1}\{i : i < \delta\}$を満たすならば} \\
			&\text{$f^{-1}(A)$は前稠密 in $P$} \} \in D_{\bar{M}}
			\end{align*}
			\item $\abs{P} > \aleph_1$かつすべての$P^\dagger \subset P$で$\abs{P^\dagger} \le \aleph_1$なものについて,$P''$であって,$\abs{P''} \le \aleph_1$かつ$P^\dagger \subset P'' \subset P$であって,$P''$は(2)の意味で$\bar{M}$-c.c.を満たし,$P'' \subsetic P$である.
		\end{enumerate}
	\end{defi}

	\begin{lem}
		定義\ref{def:oraclecc}の(2)における「ある単射$f \colon P \to \omega_1$について」は「すべての単射$f \colon P \to \omega_1$について」と変更しても同値である.
	\end{lem}
	\begin{proof}
		(2)の証拠となる単射$f \colon P \to \omega_1$を取る.
		単射$g \colon P \to \omega_1$を任意に取る.
		仮定より
		\begin{align*}
		A := \{ \delta \in \Limone : &\text{集合}A\text{が}A \in M_\delta, A \subset \delta, \text{$f^{-1}(A)$が前稠密 in $f^{-1}\{i : i < \delta\}$を満たすならば} \\
		&\text{$f^{-1}(A)$は前稠密 in $P$} \}
		\end{align*}
		は$D_{\bar{M}}$の元である.集合
		\[
		B := \{ \delta \in \Limone : f^{-1}(\{i : i < \delta\}) = g^{-1}(\{i : i < \delta\})\}
		\]
		はclubなので$B$も$D_{\bar{M}}$の元である.
		実際,$f \circ g^{-1}$と$g \circ f^{-1}$の両方で閉じている点全体の集合がclubだからである.
		集合
		\[
		C := \{ \delta \in \Limone : (\forall A \in M_\delta \cap \Pow(\delta)) ((f \circ g^{-1}) \restrict A \in M_\delta) \}
		\]
		も$D_{\bar{M}}$の元である (by 補題 \ref{lem:closure}).
		したがって,
		\begin{align*}
			\{ \delta \in \Limone : &\text{集合}A\text{が}A \in M_\delta, A \subset \delta, \text{$g^{-1}(A)$が前稠密 in $g^{-1}\{i : i < \delta\}$を満たすならば} \\
			&\text{$g^{-1}(A)$は前稠密 in $P$} \} \supseteq A \cap B \cap C \in D_{\bar{M}}
		\end{align*}
		となり証明が終わる.
	\end{proof}

	\begin{lem}[$\CH$]\label{lem:p1p2p3}
		$P_1 \subset P_2$とし,$\abs{P_1} \le \aleph_1$かつ$P_2$は可算鎖条件を満たすとする.
		このとき,$P_3$が存在して,$\abs{P_3} \le \aleph_1$かつ$P_1 \subset P_3 \lessdot P_2$を満たす.
	\end{lem}
	\begin{proof}
		$P_2$の部分集合の単調増加連続な列$\seq{P^{(\alpha)} : \alpha < \omega_1}$を次のように定める:
		まず$P^{(0)} = P_1$として,極限順序数$\alpha$については$P^{(\alpha)} = \bigcup_{\beta < \alpha} P_\beta$とおく.
		後続順序数$\alpha = \beta + 1$のときの$P^{(\alpha)}$は次のように定める.
		$P^{(\beta)}$の可算集合$I$であって$P_2$の中で前稠密ではないものの各々について,そのwitness $p_I \in P_2$を取る.$P^{(\alpha)}$は$P^{(\beta)}$に$p_I$たちを追加した集合とする.
		そして,$P^{(\alpha)}_0$の元のすべての組について,それらが$P_2$で両立するなら,共通拡大をとり,$P^{(\alpha)}_0$に追加して得られる集合を$P^{(\alpha)}$とする.
		
		最後に$P_3 = \bigcup_{\alpha < \omega_1} P^{(\alpha)}$とおけばこれが求めるべきものである.
		なお,$\CH$を仮定しているので,各ステップにおいて$I$の個数は$\aleph_1$であること,そして$P_2$が可算鎖条件を満たすので,保存すべきおのおのの極大反鎖のサイズは可算であることに注意しておく.
	\end{proof}

	\begin{lem}\label{lem:propetiesoforaclecc}
		\begin{enumerate}
			\item $P_1$と$P_2$が同型な強制概念でかつ,$P_1$が$\bar{M}$連鎖条件を満たすならば,$P_2$も$\bar{M}$連鎖条件を満たす.
			\item ある$\aleph_1$神託$\bar{M}$について$P$が$\bar{M}$連鎖条件を満たすならば,$P$は可算鎖条件を満たす.
			\item $P \lessdot Q$かつ$Q$が$\bar{M}$連鎖条件を満たすならば,$P$も$\bar{M}$連鎖条件を満たす.
			\item 定義\ref{def:oraclecc}の$\abs{P} > \aleph_1$の場合において,$P'' \lessdot P$を要求しても同値な定義となる.
		\end{enumerate}
	\end{lem}
	\begin{proof}
		(1) 明らか.
		
		(2) 濃度$\aleph_1$の場合だけ示せばほかの場合もすぐ従う.そこで$P$は台集合$\omega_1$としてよい.
		$\mathcal{J}$を濃度$\aleph_1$の極大反鎖とする.
		このときclub集合$C \subset \omega_1$があって,次を満たす:$\delta \in C$かつ$q < \delta$ならば,$p \in \mathcal{J} \cap \delta$があって$p$と両立可能である,そして,$p, q < \delta$が両立可能ならば,共通下界を$P \restrict \delta$に持つ (そういう元を割り当てる写像をとり,それで閉じている元たちからなるclub集合を取ればよい).
		
		$P$が$\bar{M}$連鎖条件を満たすので,$\delta \in C \cap I_{\bar{M}}(\mathcal{J})$が存在して以下を満たす:
		$A \in M_\delta$かつ$A$が$P \restrict \delta$の前稠密部分集合ならば,$A$は$P$の前稠密部分集合である.
		
		今,$\delta \in I_{\bar{M}}(\mathcal{J})$より$\mathcal{J} \cap \delta \in M_\delta$であり,また$\delta \in C$より$\mathcal{J} \cap \delta$は$P \restrict \delta$の前稠密集合である.
		したがって,前段落の事柄から,$\mathcal{J} \cap \delta$は$P$の前稠密集合である.
		したがって任意の$p \in \mathcal{J} \setminus \delta$はある$q \in \mathcal{J} \cap \delta$と両立可能である.
		$\mathcal{J} \setminus \delta \ne \emptyset$が$\abs{\mathcal{J}} = \aleph_1$により分かる.これは$\mathcal{J}$が反鎖なことに矛盾.
		
		(3) まず,$\abs{P} = \abs{Q} = \aleph_1$の場合を示す.
		一般性を失うことなく,$Q$の台集合は$\omega_1$としてよい.
		$\delta \in \Lim_{\omega_1}$を$Q$が$\bar{M}$連鎖条件を満たすことの証拠を与える$D_{\bar{M}}$のメンバーの元であるとする.
		集合$A$が$P \restrict \delta$で前稠密かつ$A \in M_\delta$だと仮定する.このとき$A$が$P$で前稠密であることを示さなくてはならない.
		$P \lessdot Q$なので,特に$P \subsetic Q$である.よって$A$が$Q$で前稠密であることを示せば十分である.
		$\delta$のとり方より$A$が$Q \restrict \delta$で前稠密なことを示せばよい.
		
		$q \in Q$に対して,集合$I_q$を
		\[
		I_q = \{ r \in P : \text{$r$は$q$と両立不能 または } (\forall r^\dagger \le r)(r^\dagger \in P \rightarrow \text{$r^\dagger$は$q$と両立可能} )\}
		\]
		とおく.$I_q$は$P$の稠密集合である.よって$P$の極大反鎖$J_q \subset I_q$をとれる.
		$P \lessdot Q$より$J_q$は$Q$の中でも極大反鎖である.
		$r_q \in J_q$であって,$(\forall r^\dagger \le r_q)(r^\dagger \in P \rightarrow \text{$r^\dagger$は$q$と両立可能})$なものをとる.これは取れる.なぜなら取れないとしたらすべての$r \in J_q$が$q$と両立不能なことになって,$J_q$が$Q$で極大反鎖なことに反するからである.
		今考えている$\delta$の動く範囲をあるclub集合との共通部分の中で考えることにより,$\delta$は次を満たすと仮定できる:任意の$q < \delta$に対して$r_q < \delta$,かつ$p_1, p_2 \in P \restrict \delta$が$P$で両立するなら$P \restrict \delta$でも両立する,かつ$Q$に対しても同じことが成り立つ.
		
		$A$が$Q \restrict \delta$で前稠密なことを示す.
		そのために,$q \in Q \restrict \delta$を取る.
		$r_q \in P \restrict \delta$であって$A$が$P \restrict \delta$で前稠密なので,$p \in A$があって,$p$と$r_q$は両立する.$r^\dagger \in P \restrict \delta$を$p$と$r_q$の共通拡大とする.
		$r_q$のとり方より,$r^\dagger$と$q$は両立する.
		よって$p$と$q$は両立する.したがって,$A$は$Q \restrict \delta$で前稠密であることが示された.
		これで,$\abs{P} = \abs{Q} = \aleph_1$の場合が証明された.
		
		次に一般の場合を示そう.$P$の濃度で場合分けする.
		
		$\abs{P} \le \aleph_0$のとき.このときは明らかに$P$は$\bar{M}$連鎖条件を満たす.
		
		次に$\abs{P} = \aleph_1$のとき.$\abs{Q} = \aleph_1$なら証明済みなので$\abs{Q} > \aleph_1$とする.
		$\abs{Q}$が$\bar{M}$連鎖条件を満たすことより$P^\dagger \subsetic Q$であって$P \subset P^\dagger$かつ$P^\dagger$は$\bar{M}$連鎖条件を満たし,$\abs{P^\dagger} = \aleph_1$なものが存在する.
		$P \lessdot Q$かつ$P^\dagger \subsetic Q$により,$P \lessdot P^\dagger$が分かる.
		よって,すでに示したことより$P$は$\bar{M}$連鎖条件を満たす.
		
		最後に$\abs{P} > \aleph_1$のとき.
		$P^\dagger \subset P$であって$\abs{P^\dagger} = \aleph_1$なものを任意にとる.
		補題 \ref{lem:p1p2p3}より,$P''$を見つけることができて,$\abs{P''} = \aleph_1$かつ$P^\dagger \subset P'' \lessdot P$である.
		$P \lessdot Q$なので,$P'' \lessdot Q$が従う.
		よって,小さい方の強制概念の濃度が$\aleph_1$なときの主張が証明済みなことより,$P''$は$\bar{M}$連鎖条件を満たす.ゆえに,$P$は$\bar{M}$連鎖条件を満たす.
		
		(4) $P$がもとの意味で$\bar{M}$連鎖条件を満たし,$\abs{P} > \aleph_1$であるとする.
		$P^\dagger \subset P$で$\abs{P^\dagger} \le \aleph_1$なものを任意に取る.
		補題\ref{lem:p1p2p3}により,$P''$であって,$\abs{P''} 	\le \aleph_1$かつ$P^\dagger \subset P'' \lessdot P$なものを取れる.
		(3)より$P''$は$\bar{M}$連鎖条件を満たす.これで主張が示された.
	\end{proof}

	\section{タイプの排除定理}
	
	\begin{thm}[$\diamondsuit_{\aleph_1}$]
		各$i < \omega_1$について$\psi_i(x)$を$\boldpi^1_2$論理式とする.
		$\bigwedge_{i < \omega_1} \psi_i(x)$は$V$にも$V^\C$にも解がないと仮定する.
		このとき,$\aleph_1$神託$\bar{M}$が存在して,任意の$\bar{M}$連鎖条件を満たす強制概念$P$について$\bigwedge_{i < \omega_1} \psi_i(x)$は$V^P$にも解がない.
	\end{thm}
	\begin{proof}
		自然数$n$を,強制法の定理たちが$\ZFC$の$\Sigma_n$文から証明でき,定理の仮定が$\Sigma_n$で記述できる程度に十分大きく取る.
		可算な強制概念$P$と実数の良い$P$名前$\dot{x}$に対して$(M(P, \dot{x}), \in)$を可算な$\Sigma_n$初等的な$V$の部分モデルとする.	
		\[
		I(P, \dot{x}) = \{ A \in M(P, \dot{x}) : \text{$A$は$P$の前稠密部分集合}  \}
		\]
		とおく.
		
		この状況で次の補題をまず示す.
		
		\begin{lem}\label{lem:typeomitting}
			$P \subsetic P^\dagger$かつ任意の$A \in I(P, \dot{x})$は$P^\dagger$で前稠密だとする.
			このとき
			\[
				V^{P^\dagger} \models \neg \bigwedge_{i < \omega_1} \psi_i(\dot{x}[G]).
			\]
		\end{lem}
		\renewcommand\qedsymbol{//}
		\begin{proof}
			$M(P, \dot{x})$を簡単のため$M$と書き,その推移崩壊を$N$とし,推移崩壊写像を$\pi \colon M \to N$とする.
			$G$を$(V, P^\dagger)$ジェネリックフィルターとする.
			\[
			\tilde{G} = \{ p \in \pi(P) : \pi^{-1}(p) \in G \}
			\]
			とおく.$\tilde{G}$は$(N, \pi(P))$ジェネリックフィルターである.
			
			実際,上向き閉はかんたんに示せる.
			ジェネリック性を示すために,$D \in N, D \subset \pi(P)$稠密集合を任意に取ろう.
			このとき$\pi^{-1}(D) \in I(M, \dot{x})$である.仮定より,$\pi^{-1}(D)$は$P^\dagger$で前稠密である.よって,$G$が$(V, P^\dagger)$ジェネリックなので$\pi^{-1}(D) \cap G \ne \emptyset$である.元$p \in \pi^{-1}(D) \cap G$を取る.
			このとき$\pi(p) \in D \subset$かつ$\pi^{-1}(\pi(p)) = p \in G$である.よって,$\pi(p) \in \pi(P) \cap \tilde{G}$.
			下向きに有向なことも示さないといけない.
			$p, q \in \tilde{G}$とする.
			\[ J = \{ r \in P : r \perp \pi^{-1}(p) \text{ or } r \perp \pi^{-1}(q) \text{ or } r \le \pi^{-1}(p), \pi^{-1}(q) \} \]
			とおく.
			$J \in M$かつ$J$は$P$の中で前稠密,よって仮定より$P^\dagger$の中でも前稠密である.
			ただし,$J \in M$であることを確かめるために,$M$と$V$の初等性と$P, \pi^{-1}(p), \pi^{-1}(q) \in M$と$P$が可算なので$P \subset M$であることを使った.
			したがって,$G$のジェネリック性より$J \cap G \ni r$がとれる.
			$\pi^{-1}(p), \pi^{-1}(q) \in G$より$r$について$J$の定義の前半の2つを満たすことはないので,$r \le \pi^{-1}(p), \pi^{-1}(q)$である.
			よって,$\pi(r) \in \tilde{G}$かつ$\pi(r) \le p, q$を得る.
			
			$\dot{x}$は実数の良い名前だったので,
			\[
			\dot{x} = \bigcup_{m \in \omega} \{\check{m}\} \times A_m
			\]
			で各$A_m$は$P$のantichainである.
			$\dot{x} \in M$より各$A_m \in M$を得る.
			\[
				\pi(\dot{x}) = \bigcup_{m \in \omega} \{\check{m}\} \times \pi(A_m)
			\]
			である.ここで
			\[
			\dot{x}[G] = \pi(\dot{x})[\hat{G}]
			\]
			が分かる.実際,
			\begin{align*}
			m \in \dot{x}[G] &\iff G \cap A_m \ne \emptyset \\
			&\iff \tilde{G} \cap \pi(A_m) \ne \emptyset \\
			&\iff m \in \pi(\dot{x})[\hat{G}]
			\end{align*}
			である.2つ目の同値変形については,まず上から下は$p \in G \cap A_m$を取ったとき,$P$の可算性より$p \in A_m \subset P \subset M$なので,$\pi(p) \in \tilde{G}$が分かる.$\pi(p) \in \pi(A_m)$は当たり前.よって,$\pi(p) \in \tilde{G} \cap \pi(A_m)$となる.
			下から上は順当に示せる.
			
			さて,$M$は$V$の$\Sigma_n$初等部分モデルなので,補題の仮定を満たす.
			つまり\[M \models ``\C \forces (\exists i < \omega_1)\neg \psi_i(\dot{x})"\]である.
			$M$も$P$を可算だと思っている (つまりCohen強制法だと思っている)こと,そして$M$と$N$の間の同型と強制関係の定義より,
			\[
			N[\tilde{G}] \models (\exists i < \omega_1)\neg \psi_i(\pi(\dot{x})[\tilde{G}])
			\]
			を得る.よって,$i \in N \cap \omega_1$があって,
			\[
			N[\tilde{G}] \models \neg \psi_i(\pi(\dot{x})[\tilde{G}]).
			\]
			2つの推移的モデル$N[\tilde{G}] \subset V[G]$に対して$\boldsig^1_2$上向き絶対性を使うと
			\[
			V[G] \models \neg \psi_i(\pi(\dot{x})[\tilde{G}]).
			\]
			を得る.これが得たかったことである.
		\end{proof}
		
		\renewcommand\qedsymbol{$\square$}
		
		さて,定理の証明に戻ろう.
		
		$\diamondsuit_{\aleph_1}$と閉包の議論により,次のような$\aleph_1$神託$\bar{M} = \seq{M_\delta : \delta \in \Limone}$を得ることができる:$P$が強制概念である$\delta < \omega_1$を台集合にするもので,$P, \dot{x} \in M_\delta$ならば$I(P, \dot{x}) \subset M_\delta$ ($P$と$I(P, \dot{x})$は可算である).
		この$\bar{M}$が定理の結論を満たすことを示そう.
		
		背理法で,$P^\dagger$が$\bar{M}$連鎖条件を満たし,$\dot{x}$が実数の$P^\dagger$名前で,$V^{P^\dagger} \models \bigwedge_i \psi_i(x)$であると仮定しよう.
		今,$\abs{P^\dagger} = \aleph_1$だと仮定してよい.
		なぜなら,$\abs{P^\dagger} \le \aleph_0$は定理の仮定よりありえない.
		$\abs{P^\dagger} \ge \aleph_2$のとき,補題\ref{lem:propetiesoforaclecc}より,$P''$が取れて,$\dot{x}$は$P''$名前,$\abs{P''} = \aleph_1$かつ$P'' \lessdot P^\dagger$かつ$P''$は$\bar{M}$連鎖条件を満たすものが取れる.
		$P'' \lessdot P^\dagger$なので,絶対性により$V^{P''} \models \neg\psi_i(\dot{x})$は$V^{P^\dagger} \models \neg\psi_i(\dot{x})$を導く.よって,$V^{P''} \models  \bigwedge_i \psi_i(\dot{x})$となる.したがって,背理法の仮定を満たす$P^\dagger$をサイズ$\aleph_1$で取り直すことができた.
		
		一般性を失うことなく,$P^\dagger$の台集合は$\omega_1$であるとしてよい.
		$\delta < \omega_1$であって,$P := P^\dagger \restrict \delta$が次の条件たちを満たすものを見つけられる.
		
		\begin{enumerate}
			\item $\dot{x}$は$P$名前.
			\item $P, \dot{x} \in M_\delta$.
			\item $P \subsetic P^\dagger$.
			\item $A \in M_\delta$かつ$A$が$P$で前稠密ならば$A$は$P^\dagger$でも前稠密.
		\end{enumerate}
		
		実際,$P^\dagger$が可算鎖条件を満たすことから,(1)を満たす$\delta$はcobounded manyにあり,(2)を満たす$\delta$は$D_{\bar{M}}$の定義より$D_{\bar{M}}$-manyにあり,(3)を満たす$\delta$はclub manyにある (両立元に関する閉包の議論).そして(4)を満たす$\delta$も神託連鎖条件の定義より,$D_{\bar{M}}$-manyにある.したがって,以上全部を満たす$\delta$がある.
		
		これらの事実と$\bar{M}$の構成と補題\ref{lem:typeomitting}より,$V^{P^\dagger} \models \neg \bigwedge_i \psi_i(\dot{x})$を得る.これは矛盾.
	\end{proof}

	\begin{rmk}
		補題\ref{lem:typeomitting}の証明で推移崩壊を取る必要は本当はないし,もとの本 (\cite{shelah2017proper})でも取っていない.筆者が非推移的モデルのジェネリック拡大の議論になれていないため,この形にした.
	\end{rmk}

	
	
	\nocite{*}
	\printbibliography[title={参考文献}]
	
\end{document}
